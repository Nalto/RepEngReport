
% VLDB template version of 2020-08-03 enhances the ACM template, version 1.7.0:
% https://www.acm.org/publications/proceedings-template
% The ACM Latex guide provides further information about the ACM template

\documentclass[sigconf, nonacm]{acmart}

%% The following content must be adapted for the final version
% paper-specific
\newcommand\vldbdoi{XX.XX/XXX.XX}
\newcommand\vldbpages{XXX-XXX}
% issue-specific
\newcommand\vldbvolume{14}
\newcommand\vldbissue{1}
\newcommand\vldbyear{2020}
% should be fine as it is
\newcommand\vldbauthors{\authors}
\newcommand\vldbtitle{\shorttitle} 
% leave empty if no availability url should be set
\newcommand\vldbavailabilityurl{URL_TO_YOUR_ARTIFACTS}
% whether page numbers should be shown or not, use 'plain' for review versions, 'empty' for camera ready
\newcommand\vldbpagestyle{plain} 

\usepackage{listings}
\definecolor{delim}{RGB}{20,105,176}
\definecolor{numb}{RGB}{106, 109, 32}
\definecolor{string}{rgb}{0.64,0.08,0.08}

\lstdefinelanguage{json}{
    numbers=left,
    numberstyle=\small,
    frame=single,
    rulecolor=\color{black},
    showspaces=false,
    showtabs=false,
    breaklines=true,
    postbreak=\raisebox{0ex}[0ex][0ex]{\ensuremath{\color{gray}\hookrightarrow\space}},
    breakatwhitespace=true,
    basicstyle=\ttfamily\small,
    upquote=true,
    morestring=[b]",
    stringstyle=\color{string},
    literate=
     *{0}{{{\color{numb}0}}}{1}
      {1}{{{\color{numb}1}}}{1}
      {2}{{{\color{numb}2}}}{1}
      {3}{{{\color{numb}3}}}{1}
      {4}{{{\color{numb}4}}}{1}
      {5}{{{\color{numb}5}}}{1}
      {6}{{{\color{numb}6}}}{1}
      {7}{{{\color{numb}7}}}{1}
      {8}{{{\color{numb}8}}}{1}
      {9}{{{\color{numb}9}}}{1}
      {\{}{{{\color{delim}{\{}}}}{1}
      {\}}{{{\color{delim}{\}}}}}{1}
      {[}{{{\color{delim}{[}}}}{1}
      {]}{{{\color{delim}{]}}}}{1},
}


\begin{document}
\title{RepEng Project}

%%
%% The "author" command and its associated commands are used to define the authors and their affiliations.
\author{Jonas Piehler}
\affiliation{%
  \institution{University of Passau}
  \streetaddress{Innstraße 33}
  \city{Passau}
  \state{Germany}
  \postcode{94032}
}
\email{piehlerj@fim.uni-passau.de}

%%
%% The abstract is a short summary of the work to be presented in the
%% article.
%\begin{abstract}
%Praesent imperdiet, lacus nec varius placerat, est ex eleifend justo, a vulputate leo massa consectetur nunc. %Donec posuere in mi ut tempus. Pellentesque sem odio, faucibus non mi in, laoreet maximus arcu. In hac %habitasse platea dictumst. Nunc euismod neque eu urna accumsan, vitae vehicula metus tincidunt. Maecenas %congue tortor nec varius pellentesque. Pellentesque bibendum libero ac dignissim euismod. Aliquam justo %ante, pretium vel mollis sed, consectetur accumsan nibh. Nulla sit amet sollicitudin est. Etiam ullamcorper %diam a sapien lacinia faucibus.
%\end{abstract}

\maketitle

%%% do not modify the following VLDB block %%
%%% VLDB block start %%%
\ifdefempty{\vldbavailabilityurl}{}{
\vspace{.3cm}
\begingroup\small\noindent\raggedright\textbf{Artifact:}\\
\url{https://github.com/Nalto/RepEng.git}
\endgroup
}
%%% VLDB block end %%%

\section{Introduction}

Frozza et al. \cite{SchemaExtraction} introduced a software to extract JSON Schemas from a collection of JSON or Extended JSON files.  They used five different types of documents as schown in \autoref{tab:docs} and claimed their software would be able to extract a general schema for these document types. They showed the Doc-3 JSON document in their paper, see \autoref{fig:doc3}, and based on this file, the other document types will be generated and used to confirm their claimed result shown in \autoref{fig:schema3} will be obtained after the first step of the algorithm. The order of the JSON properties will be ignored for this comparison.

\begin{table}[b]% h asks to places the floating element [h]ere.
	\caption{Input JSON Documents}
	\label{tab:docs}
	\begin{tabular}{cl}
		\toprule
		Document & Content \\
		\midrule
		Doc-1 & Base document with major JSON and JSON \\
		 & extended data types. \\
		Doc-2 & Doc-1 with changes in the data values. \\
		Doc-3 & JSON document with values representing all \\
		 & Extended JSON data types, and an array containing \\
		  & items with different data types. \\
		Doc-4 & Similar to Doc-3, but some attributes and items of \\
		 & the arrays were deleted. It was created to represent \\
		  & optional attributes. \\
		Doc-5 & Similar to Doc-4, but the attributes are in different \\
		 & order. \\
		\bottomrule
	\end{tabular}
\end{table}

\begin{figure}[b]
	\centering
	\caption{Doc-3 JSON Document}
	\label{fig:doc3}
	\lstinputlisting[language=json]{jsons/doc-3.json}
\end{figure}

\begin{figure}[b]
	\centering
	\caption{Doc-3 Raw Schema}
	\label{fig:schema3}
	\lstinputlisting[language=json]{jsons/doc-3-raw-schema.json}
\end{figure}

%\clearpage

\bibliographystyle{ACM-Reference-Format}
\bibliography{report}

\end{document}
\endinput

